\section{Discussion}
\label{sec.discussion}
In this section we will highlight the main differences between the
implementation using JSP and JSF.\\\\
JavaServer Faces provides a framework that simplifies the development and 
integration of web based user interfaces, based on MVC pattern. It is based on
component driven UI design model. It makes use of a well defined model based on managed
beans. The controller is represented by FacesServlet, which decides
the view that is required, processes events, builds
and renders the response.\\\\
JavaServer Pages is used to serve dynamically generated web pages (such as HTML
and XML).\\\\ 
For the model layer, JSF provides Managed Beans, which is a dependency injection
system.\\\\
For the controller in JSP a HttpServlet is used, which contains all the logic.
The flow of the application is determined using a parameter saved in the
request, which is then parsed. On the other hand in JSF, the FacesSevlet is
configured using a xml file, where all the necessary information is provided.
The flow of the application is established using navigation rules.
\\\\
The view in JSP provides no standard UI compoments,
everything that is displayed is standard HTML, whereas in JSF we have UI components like Button, 
Form, etc. JSF also provides event handling of the components (UI elements are statefull objects)
and it is done in Java. In JSP the event handling is done by HTTP.
